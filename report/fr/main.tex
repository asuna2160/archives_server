\documentclass[titlepage]{article}

\usepackage[utf8]{inputenc}
\usepackage[T1]{fontenc}
\usepackage{listings}
\usepackage{color}
\definecolor{mygreen}{rgb}{0,0.6,0}
\definecolor{mygray}{rgb}{0.5,0.5,0.5}
\definecolor{mymauve}{rgb}{0.58,0,0.82}
\lstset{ 
  basicstyle=\footnotesize,        % the size of the fonts that are used for the code
  breakatwhitespace=false,         % sets if automatic breaks should only happen at whitespace
  breaklines=true,                 % sets automatic line breaking
  captionpos=b,                    % sets the caption-position to bottom
  commentstyle=\color{mygreen},    % comment style
  frame=single,	                   % adds a frame around the code
  keepspaces=true,                 % keeps spaces in text, useful for keeping indentation of code (possibly needs columns=flexible)
  keywordstyle=\color{blue},       % keyword style
  language=Bash,                 % the language of the code
  numbers=left,                    % where to put the line-numbers; possible values are (none, left, right)
  numbersep=5pt,                   % how far the line-numbers are from the code
  numberstyle=\tiny\color{mygray}, % the style that is used for the line-numbers
  rulecolor=\color{black},         % if not set, the frame-color may be changed on line-breaks within not-black text (e.g. comments (green here))
  showspaces=false,                % show spaces everywhere adding particular underscores; it overrides 'showstringspaces'
  showstringspaces=false,          % underline spaces within strings only
  showtabs=false,                  % show tabs within strings adding particular underscores
  stepnumber=1,                    % the step between two line-numbers. If it's 1, each line will be numbered
  stringstyle=\color{mymauve},     % string literal style
  tabsize=2,	                   % sets default tabsize to 2 spaces
  title=\lstname                   % show the filename of files included with \lstinputlisting; also try caption instead of title
}

\begin{document}

	\title{Implémetation d'un serveur d'archives en Bash avec NetCat}
	\author{Antoine Prudhomme \\ \\ Mathilde Sandor}
	\date{\today}

	\maketitle

	\begin{abstract}
		TODO: abstract 
	\end{abstract}

	\section{Introduction}
	L'objectif du projet était d'implémenter un serveur d'archives sous Linux en Bash.
	Le serveur contient des archives, dans un format défini dans le sujet. 
	Un client peut intéragir avec le serveur au travers d'une commande shell, nommée vsh. :
	Cette commande possède 3 modes:
	\begin{itemize}  
	\item \textbf{list}, pour afficher l'ensemble des archives du serveur.
	\item \textbf{browse}, pour explorer une archive. Ce mode gère les commandes \textbf{pwd, cd, ls, rm et cat}.
	\item \textbf{extract}, qui permet d'extraire une archive sur la machine cliente.
	\end{itemize}   

	%\subsection{Subsection Heading Here}
	%Write your subsection text here.

	\section{Modèle client serveur avec NetCat}
	
	\subsection{Netcat} 
	Pour implémenter le modèle client-serveur, nous avons utilisé l'utilitaire NetCat qui permet de gérer des sockets.
	Netcat permet d'écouter sur un port réseau, de parler sur un port réseau, mais aussi au serveur de répondre aux clients en utilisant un \textbf{named pipe}.

	\subsubsection{Implémentation d'un serveur avec Netcat}

	On utilise simplement cette commande
	\begin{lstlisting}
	nc -l -p $port
	\end{lstlisting}

	Le soucis, c'est qu'une fois la reqûete traitée par le serveur, netcat ferme la connexion, donc le serveur arrête d'écouter le port. Pour résoudre ce problème, on peut simplement mettre la commande ci-dessus dans une boucle infinie: tant que le serveur tourne, on relance l'écoute du port à chaque reqûete traitée.
	\begin{lstlisting}
	while true;
	do
		echo "server is listening on port $port"
		nc -lp $port
	done
	\end{lstlisting}

	Le serveur lit sur le socket, mais on souhaite également qu'il puisse écrire sur ce même socket. Pour cela, on crée un \textbf{named pipe}, que l'on nomme ici backpipe.
	\begin{lstlisting}
	if [ ! -e backpipe ];
	then
	    mkfifo backpipe
	fi

	while true;
	do
	    echo "server is listening on port $port"
	    nc -l $port < backpipe | echo "hello world" > backpipe
	done
	\end{lstlisting}

	La sortie standard envoie les données dans le backpipe et les données sont renvoyées dans la socket.
	Le script ci dessus renvoie "hello world" à chaque requête d'un client.

	Mais le client ne se contentera pas de se connecter au serveur. Il va aussi envoyer une commande vsh au serveur (list, extract, ..). Donc le serveur doit être capable dans un premier temps d'intercepter la reqûete, puis, après avoir fait ce qu'il avait à faire, écrire une réponse dans la socket à destination du client. Le code ci dessous permet de faire ça. 

	\begin{lstlisting}
	if [ ! -e backpipe ];
	then
	    mkfifo backpipe
	fi

	while true;
	do
	    echo "server is listening on port $port"
	    req=$(nc -lp $port)
	    nc -l $port < backpipe | echo "request: $req" > backpipe
	done
	\end{lstlisting}

	On a donc un code qui permet de créer un serveur capable d'intercepter des requêtes puis de retourner une reponse au client. Il Reste donc à voir comment un client peut envoyer une requête au serveur.

	\subsection{Implémentation d'un client avec Netcat}
	Le code d'un client est beaucoup plus simple que celui d'un serveur, puisque les seuls actions qu'il a à faire sont:
	\begin{itemize}  
		\item se connecter au serveur 
		\item envoyer une requête
		\item se mettre en écoute pour attendre la réponse du serveur
	\end{itemize}

	Ci-dessous, le code du client. \textit{req.txt} est un fichier contenant la reqûete. 

	\begin{lstlisting}
	nc $host $port < req.txt
	sleep 1s
	nc $host $port
	\end{lstlisting}

	\textit{Dans notre implémentation, nous utilisons le même port pour le client et pour le serveur} 

	Nous avons donc un modèle client-serveur fonctionnel.

	\subsection{Structure de notre implémentation de VSH}
	Notre implémentation de VSH est basée sur le code client-serveur détaillé dans la section précédente.
	À la racine du projet, on trouve deux dossiers:
	\begin{itemize}  
		\item \textbf{client}, qui contient le code du client 
		\item \textbf{server}, qui contient le code du serveur
	\end{itemize}
	Ces deux dossiers sont totallement indépendants.

	\subsubsection{Structure du serveur}
	Le point d'entré du serveur est le script \textbf{server.sh}. Le code de ce script est très similaire à celui présenté plus haut. La différence est qu'ici, on ne ne renvoie pas au client sa reqûete. La reqûete du client est parsée, et à l'aide d'un \textit{case}, on décide de ce qu'il faut faire.
	\begin{itemize}  
		\item Si la requête match un cas du \textit{case} différent du cas par défaut, on execute le sous script du serveur qui correspond et on retourne au client ce que retourne ce sous script.
		\item Si la requête match le cas par défaut (donc que la reqûete n'est pas connue par le serveur) le serveur retourne un message pour indiquer que la commande est inconnue.
	\end{itemize}

	Pour lancer le serveur, il faut ouvrir un shell dans le repertoire \textbf{server} puis executer la commande suivante:

	\begin{lstlisting}
	bash server.sh $port
	\end{lstlisting}

	Les différents sous script sont détaillés dans les sections suivantes.

	\subsubsection{Structure du client}
	Le point d'entré est le script \textbf{vsh} auquel on a retiré l'extension. Ainsi, en ajoutant son repertoire à la variable \textbf{PATH} on pourra utiliser \textbf{vsh} de la même manière que les commandes shell standard (\textbf{pwd, ls, ...}). 

	\textit{Le \textbf{README.md} explique comment ajouter \textbf{vsh} à \textbf{PATH}}

	Lorsque le script \textbf{vsh} est executé, on fait un choix en fonction de la valeur du premier paramètre donné par l'utilisateur.
	\begin{lstlisting}
		Si le 1er parametre correspond a un mode vsh, alors
			Si il y a tous les arguments attendu, alors
				on execute le mode vsh
			Sinon
				on affiche un message d'erreur
		Sinon
			on affiche le contenu de usage
	\end{lstlisting}

	\section{list}

	\section{browse}

	\section{extract}

\end{document}